% Matthew Lindsay, Patrick McBryde, Michael Schultz
% CSE521S Project Proposal

\documentclass[12pt]{article}
\usepackage{fancyhdr}
\pagestyle{fancy}
\usepackage[margin=1in]{geometry}

\lhead{Matthew Lindsay, Patrick McBryde, and Michael Schultz}
\rhead{CSE 521S: Fall 2010}
\renewcommand{\headrulewidth}{0pt}
\fancyheadoffset{0.5in}

%\setlength{\evensidemargin}{0in}
%\setlength{\oddsidemargin}{0in}
%\setlength{\textwidth}{6.5in}
%\setlength{\textheight}{9.0in}
%\setlength{\topmargin}{-0.5in}
%\setlength{\headheight}{0in}
%\setlength{\headsep}{0.5in}
%\setlength{\topsep}{0in}
%\setlength{\itemsep}{0in}
%\renewcommand{\baselinestretch}{1.1}
%\newcommand{\category}[1]{ \vspace{.5in} \centerline { \bf \large \em #1 }}
%\parskip=0.020in

\begin{document}
\centerline{\bf \large Project Proposal: }

This project proposes to build parking garage monitor using a multi-hop routing
protocol using a heterogeneous mix of sensors to track parking spot
availability among other metrics.
While there exist solutions to monitor parking lot utilization, they are not as
versatile and are potentially cost prohibitive.
Our goal is to develop an extensible system that is able to provide the same
information using a variety of input sensors on an open platform.
By using an open platform, our system enables us to monitor other metrics such
as headlight, heat, or magnetic field detection with greater ease than existing
systems.
Also, unlike most existing solutions, our system will not require expensive
installation and maintenance of the hardware.

To achieve this, our project will use the Collection Tree Protocol (CTP) as a
multi-hop routing method.
This enables a single system to monitor a large area without the need to
install costly wiring or a requirement of line-of-sight communication.
To have a variety of detection methods, we intend to use the expansion
connector of the TelosB/Tmote Sky sensor motes.
By doing this we are able to swap in and out detection methods with ease.
For example, if we used a magnetometer to detect the presence of a car and the
sensor failed it would be easy to swap out the sensor for a new one or an
alternative sensing mechanism.
On the collection end, we intend to aggregate all the data into an easy to
read/interpret interface.
This is important because we need to present vehicle operators with a quick
intuitive knowledge of where they should go for the quickest parking spot to
reduce physical congestion.
By monitoring the duration of parking space use, it would also be possible for
a parking attendant to be notified when a person has remained in their space
for too long.

This project will require at least 4 TelosB/Tmote Sky sensor (1 for the base
station and 3 for detection and building an non-trivial routing topology).
We would also like a MicaZ with the MTS310CA (magnetometer) sensor board, to
test vehicle detection with a magnetometer.
Alternatively, if there is a way to interface a magnetometer with the
TelosB/Tmote Sky motes we would use that in place of the MicaZ method.

The breakdown of major components is: building the sensing devices/hardware,
writing and testing the TinyOS software to handle sensing and sending of data,
and creating a front end to present the information as needed.
If we discover any of these components is easier than others, it is easy to
combine forces to develop new/better methods for any of them.
There will also be a brief focus on finding what alternatives exist and how our
system compares to them, which we will all participate in.

% - someone has to build and test the inductor loop, which seems to be less
%   than as trivial as I thought it would be.  My attempts last night did
%   not make any thing that detects a change in induction (I have a
%   prototyping board with analog inputs and outputs among other things).
%   But it shouldn't be too hard to make one (?), or a simple metal detector
%   or something else that would detect the presence of a vehicle.  (At the
%   very least we find something so the mote doesn't have to be under the
%   car.)
%
% - someone has to build up the software to create the sensor network
%   (probably CTP) and send the data to a base station reliably.  In theory
%   this isn't too hard, but in practice I think it is harder than it should
%   be (especially testing).  The motes have buttons on them that could act
%   as the sensor input while the inductor loop doesn't exists, one press
%   says the spot if used another says the spot is free.
%
% - someone has to build a front-end that displays the number of available
%   spaces and indicates which spaces are available.  Whatever interface
%   language is picked, it should take messages from the sensor network,
%   know what sensors are in what state and display that information in the
%   best way possible (as in someone driving a car needs to look at it for a
%   split second and know where to go).
%
% - There is also some "related work" that can go into the project.
%   Wikipedia has a small page on "Parking guidance and information" systems
%   as being part of "Intelligent transportation systems."  Since the goal
%   of these things is to cause an overall reduction in traffic congestion
%   we could spend some time looking into what exists and is used.  A little
%   field work to see how e.g. the Brentwook metro parking lot tracks the
%   number of cars, the prices of such systems, etc.

\begin{itemize}
    \item Matthew:
    \item Patrick:
    \item Michael:
    \item All:
\end{itemize}

\end{document}
