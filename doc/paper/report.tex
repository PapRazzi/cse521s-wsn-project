% CSE 521S Final Report
% A Parking Guidance and Information System for TinyOS
% By Matthew Lindsay, Patrick McBryde, Michael Schultz

\documentclass{acm_proc}
\usepackage{hyperref}
\usepackage{url}
\usepackage{graphicx, subfig}

\begin{document}

\title{A Parking Guidance and Information System for TinyOS}
\subtitle{CSE 521S Final Report}

\numberofauthors{3}
\author{
\alignauthor Matthew Lindsay\\
% 2nd. author
\alignauthor Patrick McBryde\\
% 3rd. author
\alignauthor Michael Schultz\\
\and
\affaddr{Washington University in Saint Louis}\\
}

\maketitle

\begin{abstract}
``Intelligent Transportation Systems'' (ITS) are a mix of systems that
gather various data metrics from transportation areas (parking lots,
streets, alleyways, etc.), aggregate this data together, and present
coherent information to the end-users of the system.
Orchestrating such a system presents several challenges at each level.
How do you gather the data and at what granularity?
Where does the data go once gathered and can it be useful to anyone?
If it can be useful, how can it be presented to end-users to provide
accurate and simple-to-interpret content?
This paper presents and explains our decisions in developing a ``Parking
Guidance and Information'' (PGI) system for TinyOS.
\end{abstract}

\section{Introduction}

% From proposal
This project proposes to build parking guidance and information system on
the
TinyOS platform.
We will take advantage of the multi-hop routing abilities and a
heterogeneous
mix of sensors to keep track of parking spot availability and usage data,
possible using other sensing metrics as needed.
In the end, we plan to have developed an extensible system able to provide
the
same information as existing systems with the addition of being able to
support different sensors by using an open platform.
TinyOS will allow us to extend our monitoring to other metrics such as
headlight, heat, or magnetic field detection with great ease.
Also, unlike most existing solutions, our system will not require expensive
installation and simplifies hardware and software expansion.
% end from proposal


\cite{sakai:pgi-toyota}
\cite{vincent:streetview}
\cite{pgi:signal-park}
\cite{pgi:streetline}
PGI systems are designed to keep track of vehicular information in parking
garages, at present the most common form simply counts the number of spaces
available in a designated area and can be prohibitively expensive.

More details text about what this is about, what exists, why we wanted to
do this (why it was interesting/motivation, etc.).

What the rest of the paper has in store.

\section{Goals}\label{sec:goals}

What are the goals for this project, what we set out to do (in detail).

% from proposal
To achieve these goals, we intend to use the Tmote Sky/TelosB mote platform
which have an expansion connector that allows external sensors to be
connected.
To reduce installation costs, we will use a multi-hop routing protocol
(likely
the collection tree protocol) to communicate data from a sensor to the base
station.
This allows for a large number of sensors to be deployed with little effort
or
physical infrastructure to be in place.
Finally, once the data is aggregated at a central location it must be
displayed
to the end user in an easy to read/interpret interface.
This is important because the vehicle operator needs to have a quick,
intuitive
knowledge of where to go, so they don't cause physical congestion.
We also wish to monitor the duration of parking space use, to allow a
parking
attendant to be notified when a person has overused their space.
% end from proposal

\section{Design}\label{sec:design}

Design and components of the infrastructure.  What we need to succeed in
out goals.
High level description of the various components we developed, to be
explained in detail below.
Probably a graphics of the gist of the system.

\subsection{Hardware}

This should cover the physical hardware we used in building this.
Including (but not limited to):

- TelosB motes (listing any relevant specifications, why we used TelosB
 and not something else) \cite{xbow:telosb-datasheet}.

- External sensor we used, specifications for that, more detail than
 TelosB since the reader will be less familiar with it, why we used that.

At least one pictures of the telosb, the sensors, or the telosb with sensor
attached.

% from proposal
We would like 6 TelosB/Tmote Sky sensors (2 for sensor testing and 4 for
network
development to build a non-trivial routing topology).
It would also be interesting to use a Mica family board with the MTS310CA
(magnetometer) sensor board, to test vehicle detection with a magnetometer,
though not strictly required as we can find other sensors to use.
Alternatively, if we can interface a magnetometer with the TelosB/Tmote Sky
mote board that could be used instead of a Mica family board.
% end from proposal

\subsection{Software}

Yup we had software.

A paragraph or two about TinyOS and building whatever we used there
(interfaces, interaction with other sensors and base
station, CTP)~\cite{tep119:collection}.

A paragraph about the base station software and what it does.

A paragraph laying out what Google AppEngine is, why we chose it, and what
it does for us.

A paragraph or two describing the AE backend and frontend(s).

\section{Implementation}

More detailed description of the underlying implementation (packet formats
in the network, sending data to AppEngine, specific technology used on
backend, any frontend stuff).

% from proposal
The major components of this project are building and developing the
sensing
devices, writing and testing the networking and communications
software to handle data delivery, and creating a friendly front end to
present
and track the information as needed.
If we discover any of these components are significantly easier than
others, it
is easy to combine forces to develop new/better methods for any of them.
We also have an interest in discovering more specific information of
existing
systems and getting up-to-date in the area of intelligent transportation
systems.
% end from proposal

\section{Experiment}\label{sec:experiment}

Paragraph about experiments that show our system works.

\section{Related Works}\label{sec:related}

\section{Lessons Learned}\label{sec:lessons}

Anything we would have done differently if we were to pursue this project
in more detail again.

\section{Conclusions and Future Work}\label{sec:conclusions}

Conclusions from this project.

% \section{References}
\bibliographystyle{abbrv}
\bibliography{report}

\end{document}
